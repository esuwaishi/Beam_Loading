\documentclass[article]{jlreq}

\usepackage{graphicx}
\usepackage{amsmath,amssymb,amsthm}
%\usepackage{mathtools}
%\usepackage{siunitx}
\usepackage{physics}
\usepackage{bm}

\usepackage{ulem}

% \usepackage[pdfencoding=auto]{hyperref}
% \usepackage{xcolor}
% \hypersetup{
%   bookmarksnumbered=true,
%   colorlinks=true,
%   citecolor=red,
%   linkcolor=blue,
%   urlcolor=orange,
% }

\renewcommand{\today}{\the\year/\the\month/\the\day}
\renewcommand{\contentsname}{Contents}
\renewcommand{\abstractname}{Abstract}
\renewcommand{\refname}{References}
\renewcommand{\figurename}{Fig.~}
\renewcommand{\tablename}{Table~}

%\AtBeginDocument{\RenewCommandCopy\qty\SI}

\begin{document}
\title{AM to PM conversion of linear filters}
\author{Magnus Danielson}
\maketitle
%\tableofcontents
%%\clearpage

\begin{abstract}
    変調信号がフィルターを通過する際の振幅変調と位相変調の間の変換が分析されます。変調された側波帯の振幅がフィルタをどのように通過するか、および AM と PM が側波帯の 1つで反対の符号をどのように相互作用するかの違い。AM と PM の間の変換がモデル化され、散乱モデルと線形フィルター伝達関数に基づく2つの関数の評価が提供されます。システム帯域幅の影響が分析され、AM と PM の分離を確実にするための経験則が開発されます。
\end{abstract}

\section{INTRODUCTION}
%

%
\begin{thebibliography}{9}
    \bibitem{ref:1}
    Eva S. Ferre-Pikal, Fred L. Walls and Craig W. Nelson, Guidelines for Designing BJT Amplifiers with Low 1/f AM and PM noise, April 1997 IEEE Transactions on Ultrasonics Ferroelectrics and Frequency Control 44(2):335 - 343, DOI10.1109/58.585118
    B.P. Lathi, Signals, systems and communication, Publ. John Wiley \& Sons, 1965.
    \bibitem{ref:2}
    L. Rde and B. Westergren, Mathematics Handbook for Science and Engineering, 5th ed., Lund, Sweden: Studentliteratur, 2010.
\end{thebibliography}
%
%
\end{document}